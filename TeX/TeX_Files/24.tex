\section{Подгруппа азота. Типичные степени окисления. Строение простых веществ. Водородные соединения.
Получение и свойства аммиака, соли аммония. Кислородные кислоты азота и фосфора.}
\subsection{Подгруппа азота}
\begin{wrapfigure}[24]{r}{0.2\textwidth}
\centering
\begin{tikzpicture}
\filldraw[fill=red!22!white] (-0.7,0.5) -- (-0.7,-0.5) -- (0.7,-0.5) -- (0.7,0.5) -- cycle;
\node[] at (-0.4,0.2) {\ce{N}};
\node[] at (0.4,0.2) {7};
\node[] at (0.0, -0.25) {\tiny Азот};
\filldraw[fill=red!22!white,yshift=-1.1cm] (-0.7,0.5) -- (-0.7,-0.5) -- (0.7,-0.5) -- (0.7,0.5) -- cycle;
\node[yshift=-1.1cm] at (-0.4,0.2) {\ce{P}};
\node[yshift=-1.1cm] at (0.4,0.2) {15};
\node[yshift=-1.1cm] at (0.0, -0.25) {\tiny Фосфор};
\filldraw[fill=yellow!22!white,yshift=-2.2cm] (-0.7,0.5) -- (-0.7,-0.5) -- (0.7,-0.5) -- (0.7,0.5) -- cycle;
\node[yshift=-2.2cm] at (-0.4,0.2) {\ce{As}};
\node[yshift=-2.2cm] at (0.4,0.2) {33};
\node[yshift=-2.2cm] at (0.0, -0.25) {\tiny Мышьяк};
\filldraw[fill=yellow!22!white,yshift=-3.3cm] (-0.7,0.5) -- (-0.7,-0.5) -- (0.7,-0.5) -- (0.7,0.5) -- cycle;
\node[yshift=-3.3cm] at (-0.4,0.2) {\ce{Sb}};
\node[yshift=-3.3cm] at (0.4,0.2) {51};
\node[yshift=-3.3cm] at (0.0, -0.25) {\tiny Сурьма};
\filldraw[fill=blue!22!white,yshift=-4.4cm] (-0.7,0.5) -- (-0.7,-0.5) -- (0.7,-0.5) -- (0.7,0.5) -- cycle;
\node[yshift=-4.4cm] at (-0.4,0.2) {\ce{Bi}};
\node[yshift=-4.4cm] at (0.4,0.2) {83};
\node[yshift=-4.4cm] at (0.0, -0.25) {\tiny Висмут};
\filldraw[fill=blue!22!white,yshift=-5.5cm] (-0.7,0.5) -- (-0.7,-0.5) -- (0.7,-0.5) -- (0.7,0.5) -- cycle;
\node[yshift=-5.5cm] at (-0.4,0.2) {\ce{Mc}};
\node[yshift=-5.5cm] at (0.35,0.2) {115};
\node[yshift=-5.5cm] at (0.0, -0.25) {\tiny Московий};
\end{tikzpicture}
\caption{Подгруппа азота. Красные -- выраженные неметаллы, жёлтые -- полуметаллы, синие -- металлы.}
\label{fig:azotgruop}
\end{wrapfigure}
В подгруппу азота входят сам азот, фосфор, мышьяк, сурьма, висмут и московий (последний мало интересен с точки зрения химии, т.к. период его полураспада даже для самых стабильных ядер составляет несколько десятков милисекнд) см. рис.~\ref{fig:azotgruop}. 
\begin{itemize}
    \item[\textbf{Азот}] -- бесцветный газ, не имеющий запаха  (при нормальных условиях), безвреден, не поддерживает дыхание и горение, мало растворим в воде. Азот в форме \ce{N2} состаляет больше 70\% земной атмосферы.

    Также может быть и в жидком состоянии, при температуре кипения ($−195.8 {}^{\circ}\mathrm{C}$) -- бесцветная жидкость. При контакте с воздухом поглощает кислород.

    При температуре в $−209.86 {}^{\circ}\mathrm{C}$ азот переходит в твердое состояние в виде снега. При контакте с воздухом поглощает кислород, при этом плавится, образуя раствор кислорода в азоте.
    
    \textbf{Химически свойчтва азота}:
    В свободном состояние существует в виде молекул \ce{N2}, со структурной формулой \ce{N#N}. При нормальных условиях практически не диссоциирует. Вследствие большой прочности молекулы азота некоторые его соединения эндотермичны (многие галогениды, азиды, оксиды), то есть энтальпия их образования положительна, а соединения азота термически малоустойчивы и довольно легко разлагаются при нагревании. Именно поэтому азот на Земле находится по большей части в свободном состоянии. Ввиду своей значительной инертности азот при обычных условиях реагирует только с литием:
    \begin{equation*}
        \ce{3Li + N2 -> 2Li3N}  
    \end{equation*}
    Поулучить азот можно нагревая нитрит аммония:
    \begin{equation*}
        \ce{NH4NO3 ->[t^{\circ}] 2H2O + N2 ^} 
    \end{equation*}
    Однако при этом азот ооказывается загрязнён и требует очистки. В промышленности его обычно получают из воздуха. Азот черезвычайно важен для промышленного получения аммиака.
    \item[\textbf{Фосфор}] -- в нормальных условиях фосфор существует в несколькоих аллотропных формах:
    \begin{enumerate}
        \item \textbf{Белый фосфор} -- очень активное, высокотоксичное, твёрдое, белое вещество, напоминающее парафин. Белый фосфор имеет молекулярную кристаллическую решётку, формула молекулы белого фосфора — \ce{P4}, причём атомы расположены в вершинах тетраэдра. Плохо растворяется в воде[6], но легкорастворим в органических растворителях.
        
        Химически белый фосфор чрезвычайно активен. Например, он медленно окисляется кислородом воздуха уже при комнатной температуре и светится (бледно-зелёное свечение). Явление такого рода свечения вследствие химических реакций окисления называется хемилюминесценцией (иногда ошибочно фосфоресценцией). При взаимодействии с кислородом белый фосфор горит даже под водой. 
        
        \item \textbf{Жёлтым фосфором} называют загрязнённый белый фосфор.
        \item \textbf{Красный фосфор} — это более термодинамически стабильная модификация элементарного фосфора, которая имеет формулу \ce{Р_{n}} и представляет собой полимер со сложной структурой. В зависимости от способа получения и степени дробления, красный фосфор имеет оттенки от пурпурно-красного до фиолетового, а в литом состоянии — тёмно-фиолетовый с медным оттенком, имеет металлический блеск. Химическая активность красного фосфора значительно ниже, чем у белого; ему присуща исключительно малая растворимость. Растворить красный фосфор возможно лишь в некоторых расплавленных металлах (свинец и висмут), чем иногда пользуются для получения крупных его кристаллов. Не самовоспламеняется и не люменисцирует. Красный фосфор не ядовит, используется при изготовлении тёрок спичечных коробков.
        \item \textbf{Чёрный фосфор} -- то наиболее стабильная термодинамически и химически наименее активная форма элементарного фосфора. Впервые чёрный фосфор был получен в 1914 году американским физиком П. У. Бриджменом из белого фосфора в виде чёрных блестящих кристаллов, имеющих высокую ($2690 \text{кг}/\text{м}^{3}$) плотность. Для проведения синтеза чёрного фосфора Бриджмен применил давление в $2\cdot10^{9}$ Па (20 тысяч атмосфер) и температуру около $200 {}^{\circ}\mathrm{С}$. Начало быстрого перехода лежит в области 13 000 атмосфер и температуре около $230 {}^{\circ}\mathrm{С}$.

        Чёрный фосфор представляет собой чёрное вещество с металлическим блеском, жирное на ощупь и весьма похожее на графит, и с полностью отсутствующей растворимостью в воде или органических растворителях. Поджечь чёрный фосфор можно, только предварительно сильно раскалив в атмосфере чистого кислорода до $400 {}^{\circ}\mathrm{С}$. Чёрный фосфор проводит электрический ток и имеет свойства полупроводника. Температура плавления чёрного фосфора $1000 {}^{\circ}\mathrm{С}$ под давлением $1.8\cdot10^{6}$ Па.
    \end{enumerate}
    \textbf{Химические свойства фосфора}: Легко реагирует с кислородом:
    \begin{gather*}
        \ce{4P + 5O2 -> 2P2O5} \ \text{(с избытком кислорода)}\\
        \ce{4P + 3O2 -> 2P2O3} \ \text{(с недостатком кислорода)}
    \end{gather*}
    С металлами — окислитель, образует фосфиды:
    \begin{equation*}
        \ce{2P + 3Ca -> Ca3P2}
    \end{equation*}
    C неметаллами — восстановитель:
    \begin{equation*}
        \ce{2P + 5Cl2 -> 2PCl5}
    \end{equation*}
    Фосфиды разлагаются водой и кислотами с образованием фосфина (газа аналогичного аммиаку -- \ce{PH3}).
    Сильные окислители превращают фосфор в фосфорную кислоту:
    \begin{equation*}
        \ce{2P + 5H2SO4 -> 2H3PO4 + 5SO2 + 2H2O}
    \end{equation*}
    \item[\textbf{Мышьяк}] -- как простое вещество представляет собой хрупкий полуметалл стального цвета с зеленоватым оттенком (в серой аллотропной модификации). Яд и канцероген.
    \item[\textbf{Сурьма}] --   Простое вещество сурьма — полуметалл серебристо-белого цвета с синеватым оттенком, грубозернистого строения. Известны четыре металлических аллотропных модификаций сурьмы, существующих при различных давлениях, и три аморфные модификации (взрывчатая, чёрная и жёлтая сурьма).
    \item[\textbf{Висмут}] -- Простое вещество представляет собой при нормальных условиях блестящий серебристый с розоватым оттенком металл.
    \end{itemize}
     Элементы главной подгруппы V группы имеют пять электронов на внешнем электронном уровне. В целом характеризуются как неметаллы. Способность к присоединению электронов выражена значительно слабее, по сравнению с халькогенами и галогенами. Все элементы подгруппы азота имеют электронную конфигурацию внешнего энергетического уровня атома $n\mathrm{s}^{2}n\mathrm{p}^{3}$ и могут проявлять в соединениях степени окисления от $−3$ до $+5$. Вследствие относительно меньшей электроотрицательности связь с водородом менее полярна,чем связь с водородом халькогенов и галогенов. Водородные соединения этих элементов не отщепляют в водном растворе ионы водорода, иными словами, не обладают кислотными свойствами. Первые представители подгруппы — азот и фосфор — типичные неметаллы, мышьяк и сурьма проявляют металлические свойства, висмут — типичный металл. Таким образом, в данной группе резко изменяются свойства составляющих её элементов: от типичного неметалла до типичного металла. Химия этих элементов очень разнообразна и, учитывая различия в свойствах элементов, при изучении её разбивают на две подгруппы — подгруппу азота и подгруппу мышьяка. Редко используемое альтернативное название этой группы элементов - пниктогены, в переводе с греческого языка означает удушающий, что больше относилось к первому элементу группы - азоту, который, несмотря на безвредность, не поддерживает горения и дыхания. Однако данное название в целом хорошо характеризует данную группу элементов, так как большинство из них, как в виде простого вещества, так и в виде соединений очень ядовиты.
     \subsection{Аммиак}
     \textbf{Аммиак} (нитрид водорода, аммониак, гидрид азота) — бинарное неорганическое химическое соединение азота и водорода с формулой \ce {NH3}, при нормальных условиях — бесцветный газ с резким характерным запахом.
     \subsubsection{Получение аммиака}
     В промышленности аммиак получается за счёт процесса Габера:
     \begin{equation*}
        \ce{N2 + 3H2 -> 2NH3 ^} + Q 
     \end{equation*}
     Выход аммиака зависит от условий проведения реакции, для большого выхода следует выдержать верный баланс высоких давлений и высоких температур.
     
     В лаборатории аммиак можно получать другими способами, например:
     \begin{gather*}
         \ce{NH4Cl + NaOH -> NaCl + H2O + NH3 ^}\\
         \ce{2NH4Cl + Ca(OH)2 -> CaCl2 + 2H2O + 2NH3 ^}
     \end{gather*}
     \subsubsection{Химические свойства аммиака. Аммоний. Соли аммония. Амиды.}
     Присоединяя протон аммиак образует ион аммония \ce{NH4^{+}}. Водный раствор аммиака («нашатырный спирт») имеет слабощелочную реакцию из-за протекания процесса:
     \begin{equation*}
         \ce{NH3 + H2O -> NH4^{+} + OH^{-}}
     \end{equation*}
     Взаимодействуя с кислотами, даёт соответствующие соли аммония:
    \begin{gather*}
        \ce{NH3 + HNO3 -> NH4NO3}\\
        \ce{NH3 + HCl -> NH4Cl}\\
        \ce{2NH3 + H2SO4 -> (NH4)2SO4}
    \end{gather*}
    Аммиак также способен образовывать с металлами соли — амиды, имиды и нитриды. Соединения, содержащие ионы \ce {NH2^-}, называются амидами, \ce{NH^{2-}} -- имидами, а \ce {N^{3-}} -- нитридами. Амиды щелочных металлов получают, действуя на них аммиаком:
    \begin{equation*}
        \ce{2NH3 + 2K -> 2KNH2 + H2 ^}
    \end{equation*}
    Амиды металлов являются аналогами гидроксидов. Эта аналогия усиливается тем, что ионы \ce{OH^{-}} и \ce{NH2^{-}}, а также молекулы \ce{H2O} и \ce{NH3} изоэлектронны. Амиды являются более сильными основаниями, чем гидроксиды, а следовательно, подвергаются в водных растворах необратимому гидролизу:
    \begin{equation*}
        \ce{NaNH2 + H2O -> NaOH + NH3}
    \end{equation*}
    Галогены (хлор, йод) образуют с аммиаком опасные взрывчатые вещества — галогениды азота (хлористый азот, иодистый азот).
