\section{Окислительно-восстановительные потенциалы. Сопряженные окислители и восстановители. Уравнение Нернста. Диаграммы Латимера. Связь ЭДС с термодинамическими свойствами.}

\subsection{Окислительно-восстановительные потенциалы.}
Окислительно-восстановительный потенциал ($E_{h}$) -- мера способности химического вещества присоединять электроны. Измеряется в В. Величина $E_{h}$ зависит от изменения в растворе концентраций \ce{Н^{+}} и \ce{ОН^{-}} ионов (т. е. от степени кислотности или щелочности среды), от соотношения концентрации окисляющего и восстанавливающегося ионов и от температуры. %Eh, при котором концентрация окислителя равна концентрации восстановителя (напр., Fe2+ = Fe3+), называется стандартным. 

Если ОВП принимает отрицательные значения, то это вещество имеет восстановительные свойства, если положительные -- окислительные.

Окислительно-восстановительный потенциал определяют как электрический потенциал, устанавливающийся при погружении инертного электрода в окислительно-восстановительную среду, то есть в раствор, содержащий как восстановленное соединение (\ce{A_{red}}), так и окисленное соединение (\ce{A_{ox}}). Если полуреакцию восстановления представить уравнением:
\begin{equation*}
\ce{A_{ox}} + n\cdot e^{-} \ce{-> A_{red}}},
\end{equation*}
то количественная зависимость окислительно-восстановительного потенциала от концентрации (точнее активностей) реагирующих веществ выражается уравнением Нернста.

\subsection{Сопряженные окислители и восстановители.}
\textbf{Окисление} -- процесс отдачи электронов с увеличением степени окисления.
Восстановитель, отдавая электроны, приобретает окислительные свойства, превращаясь в сопряжённый окислитель (сам процесс называется окислением)

\textbf{Восстановление} -- процесс присоединения электронов атомом вещества, при этом его степень окисления понижается.
Окислитель, принимая электроны, приобретает восстановительные свойства, превращаясь в сопряжённый восстановитель (сам процесс называют восстановлением)

Окислитель и его восстановленная форма, либо восстановитель и его окисленная форма составляет сопряжённую окислительно-восстановительную пару, а их взаимопревращения являются окислительно-восстановительными полуреакциями.

\subsection{Уравнения Нернста.}

Уравнение Нернста -- уравнение~(\ref{eq:nernst}), связывающее окислительно-восстановительный потенциал системы с активностями веществ, входящих в уравнение, и стандартными электродными потенциалами окислительно-восстановительных пар. 
\begin{equation}
    \label{eq:nernst}
    E_{h} = E^{\circ} + \frac{RT}{nF}\cdot\ln\left(\prod_{i}a_{i}^{\nu_{i}}\right),
\end{equation}
где $E^{\circ}$  -- стандартный электродный потенциал (табличное значение), $R$ -- универсальная газовая постоянная, $T$ -- абсолютная температура, $F$ -- постоянная Фарадея, $n$ -- число электронов, участвующих в процессе, $a_{i}$ --  активности участников полуреакции, $\nu_{i}$ -- их стехиометрические коэффициенты (положительны для продуктов полуреакции (окисленной формы), отрицательны для реагентов (восстановленной формы)). В простейшем случае:
\begin{gather*}
    \ce{Red} - ne^{-} \ce{<=> Ox} \\
    E_{h} = E^{\circ} + \frac{RT}{nF}\cdot\ln\left(\frac{a_{\ce{Ox}}}{a_{\ce{Red}}}\right) \approx E^{\circ} + \frac{RT}{nF}\cdot\ln\left(\frac{C_{\ce{Ox}}}{C_{\ce{Red}}}\right)
\end{gather*}
Здесь $a_{\ce{Ox}}$ и  $C_{\ce{Ox}}$ -- активность и концентрация окислителя,  $a_{\ce{Red}}$ и  $C_{\ce{Red}}$ -- активность и концентрация восстановителя.

4. Диаграммы Латимера 

Диаграмма Латимера в сокращенном виде представляет стандартные электродные потенциалы между различными формами одного элемента с разными степенями окисления. 
